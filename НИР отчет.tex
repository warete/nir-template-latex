\documentclass[14pt,a4paper]{article}
\usepackage[T2A]{fontenc}
\usepackage[utf8]{inputenc}
\usepackage{amssymb}
\usepackage[russian]{babel}
%\renewcommand{\rmdefault}{ftm}  %Times New Roman
%\usepackage{graphicx}
%\usepackage{courier}
\usepackage{amsmath}


\usepackage{graphicx}
\graphicspath{{images/}}
\DeclareGraphicsExtensions{.eps,.bmp,.jpg,.png}
%\usepackage{ccaption}
%\captiondelim{~---~} 
%\addto\captionsrussian{ \def\figurename{Рисунок} }
\usepackage{caption}
\captionsetup[figure]{name=Рисунок,labelsep=endash,font={Large}}
\usepackage{listings}
%\usepackage[font={Large}]{caption}

\usepackage{listings}
\usepackage{algorithm2e}

%\lstset{numbers=left , language= C  }

\usepackage{tocloft}
%mcourier
%\renewcommand{\ttdefault}{fcr}

\usepackage{geometry}
\geometry{left=3.0cm}
\geometry{right=1.0cm}
\geometry{top=2cm}
\geometry{bottom=2cm}

\newcommand{\biggI}{\renewcommand{\baselinestretch}{1.3}}
% команда \biggI для полуторного межстрочного интервала
\makeatletter
\renewcommand{\@oddfoot}{\hfil \Large \arabic{page}\hfil}
\renewcommand{\@evenfoot}{\hfil \Large \arabic{page}\hfil}
\makeatother
\renewcommand{\thepage}{\hfil \Large \arabic{page}\hfil}

\usepackage{indentfirst} % Красная строка

\fontencoding{T1}
\fontfamily{pcr}
\selectfont


 %---------------- заголовки ----------- %
\usepackage{titlesec}
\usepackage{enumerate}
\titleformat{\chapter}[display]
    {\filcenter}
    {\MakeUppercase{\chaptertitlename} \thechapter}
    {8pt}
    {\bfseries}{}
 
\titleformat{\section}[block]
    {\Large\bfseries\sloppy\righthyphenmin62}
    {\thesection}
    {1em}{}
 
\titleformat{\subsection}[block]
    {\Large\bfseries\sloppy\righthyphenmin62}
    {\thesubsection}
    {1em}{}

\titleformat{\subsubsection}[block]
    {\Large\bfseries\sloppy\righthyphenmin62}
    {\thesubsubsection}
    {1em}{}


\usepackage{titletoc}
%\titlecontents{section}[1.6em]{}{1.6em}{1pc}
%\titlecontents{subsection}[2.6em]{}{2.6em}{1pc}
%\titlecontents{subsubsection}[2.6em]{}{2.6em}{1pc}
\titlecontents{section}
[2.6em] % ie, 1.5em (chapter) + 2.3em
{\sloppy\righthyphenmin62}
{\contentslabel{2.6em}}
{\hspace*{-2.6em}}
{\titlerule*[1pc]{}\contentspage}

\titlecontents{subsection}
[2.6em] % ie, 1.5em (chapter) + 2.3em
{\sloppy\righthyphenmin62}
{\contentslabel{2.6em}}
{\hspace*{-2.6em}}
{\titlerule*[1pc]{}\contentspage}

\titlecontents{subsubsection}
[2.6em] % ie, 1.5em (chapter) + 2.3em
{\sloppy\righthyphenmin62}
{\contentslabel{2.6em}}
{\hspace*{-2.6em}}
{\titlerule*[1pc]{}\contentspage}


    
% Настройка вертикальных и горизонтальных отступов
% Настройка вертикальных и горизонтальных отступов
\titlespacing{\section}{\parindent}{15mm}{10mm}
\titlespacing{\subsection}{\parindent}{15mm}{10mm}
\titlespacing{\subsubsection}{\parindent}{15mm}{10mm}

%------- конец заголовков --------- %


\lstdefinelanguage{cs}
  {morekeywords={abstract,event,new,struct,as,explicit,null,switch
		base,extern,object,this,bool,false,operator,throw,
		break,finally,out,true,byte,fixed,override,try,
		case,float,params,typeof,catch,for,private,uint,
		char,foreach,protected,ulong,checked,goto,public,unchecked,
		class,if,readonly,unsafe,const,implicit,ref,ushort,
		continue,in,return,using,decimal,int,sbyte,virtual,
		default,interface,sealed,volatile,delegate,internal,short,void,
		do,is,sizeof,while,double,lock,stackalloc,
		else,long,static,enum,namespace,string, },
	  sensitive=false,
	  morecomment=[l]{//},
	  morecomment=[s]{/*}{*/},
	  morestring=[b]",
}

\setlength\cftaftertoctitleskip{11mm}
\renewcommand{\cftdot}{}
  
%\setmainfont{Times New Roman}
\parindent=1.25cm % абзацный отступ
\biggI


\thispagestyle{empty} % не ставим номер страницы
\usepackage{float}%"Плавающие" картинки
\newcommand{\imgh}[3]{
\begin{figure}[H]
\center{
\includegraphics[width=#1]{#2}}
\caption{#3}
\label{ris:#2}
\end{figure}
}

\makeatletter
\bibliographystyle{utf8gost705u} % Оформляем библиографию в соответствии с ГОСТ 7.0.5
\renewcommand{\@biblabel}[1]{#1.}   % Заменяем библиографию с квадратных скобок на точку:
\makeatother

% новая команда \RNumb для вывода римских цифр
\newcommand{\RNumb}[1]{\uppercase\expandafter{\romannumeral #1\relax}}

%  маркированные списки
\renewcommand{\labelitemi}{-}
\renewcommand{\labelitemii}{-}
%  нумерованные списки
\renewcommand{\labelenumi}{\arabic{enumi})}
\usepackage{enumitem}
\setlist{nosep, leftmargin=\parindent}

%\usepackage{hyperref}
%\urlstyle{same}

\begin{document}
\renewcommand*\contentsname{\parindent=1.25cm Содержание}
\Large
\setcounter{page}{5} % начать нумерацию с номера три

\newpage
\tableofcontents % это оглавление, которое генерируется автоматически
\Large
\setcounter{secnumdepth}{-1}

\input{introdution}

\setcounter{secnumdepth}{3}

\newpage
\section{\Large Глава}\vspace{-7mm}
\subsection{Подглава}
Текст
%------------------------------------------------------------------------------------------------------
%------------------------------------------------------------------------------------------------------


\newpage
\section{\Large Глава}
Текст.
\subsection{Подглава}
Текст

%------------------------------------------------------------------------------------------------------
%------------------------------------------------------------------------------------------------------


\newpage
\section{\Large Разработка программы для пакетной обработки температурных данных}
Текст (рисунок ~\ref{ris:image_500x400})
\\
\imgh{1\linewidth}{image_500x400}{Подпись к картинке}
\par
Текст (рисунок ~\ref{ris:image_600x300})
\\
\imgh{1\linewidth}{image_600x300}{Подпись к картинке}

%------------------------------------------------------------------------------------------------------

\subsection{Подглава}
Текст:
\begin{itemize}
	\item[-] Элемент списка;
	\item[-] Элемент списка;
	\item[-] Элемент списка;
	\item[-] Элемент списка;
	\item[-] Элемент списка.
\end{itemize}
\par
Текст

%------------------------------------------------------------------------------------------------------
\subsection{Подглава}
Текст (таблица 1).
\begin{table}[H]
	\begin{flushleft}\hspace{1.25cm}\Large Таблица 1 -- \label{tab:exp2values}Заголовок таблицы\end{flushleft}
	\begin{center}
		{\Large
			\begin{tabular}{|c|c|c|c|}
				\hline
				\hspace{1cm}$x$\hspace{1cm} & \hspace{1cm}$y$\hspace{1cm} & \hspace{1cm}$z$\hspace{1cm} & \hspace{1cm}$T$\hspace{1cm} \\
				\hline
				$x_1$ & $y_1$ & $z_1$ & $T_1$ \\
				\hline
				$x_2$ & $y_2$ & $z_2$ & $T_2$ \\
				\hline
				... & ... & ... & ... \\
				\hline
				$x_n$ & $y_n$ & $z_n$ & $T_n$ \\
				\hline
			\end{tabular}
		}
	\end{center}
\end{table}
\par
Текст

%------------------------------------------------------------------------------------------------------

\subsection{Подглава}\vspace{-7mm}
\subsubsection{Параграф}
Текст (листинг 1).
\par Листинг 1 -- Подпись к листингу\\
\large
\begin{verbatim}
class Program
{
    static void Main(string[] args)
    {
        Thread threadA = new Thread(() => Console.WriteLine("ThreadA"));
        Thread threadB = new Thread(() => Console.WriteLine("ThreadB"));
        threadA.Start();
        threadB.Start();
    }
}
\end{verbatim}
\Large
\par
Текст

\subsubsection{Параграф}
Текст

%------------------------------------------------------------------------------------------------------
%------------------------------------------------------------------------------------------------------


\setcounter{secnumdepth}{-1}
\input{conclusion}

\newpage
\addcontentsline{toc}{subsection}{Список литературы}
\begin{thebibliography}{3}
	\bibitem{bardati} Bardati, F. Modeling the Visibility of Breast Malignancy by a MicrowaveRadiometer [Текст] / F. Bardati, S. Iudicello. — Biomed. Engineering. -- 2008. -- Vol.55 (6). -- С. 214-221.
	
	\bibitem{cSharp3} Albahari, B. C\# 4.0 in a Nutshell [Текст] / B. Albahari, J. Albahari . -- O'Reilly Media. -- 4th Edition. -- 2010. -- 1058 С.
	
	\bibitem{pryor2011} Pryor, R.W. Multiphysics Modeling Using COMSOL: A First Principles Approach [Текст] / R.W. Pryor -- Jones \& Barlett Publishers, Inc. -- 2011. -- 872 с.
	
	\bibitem{Roger2011} Roger, W. Multiphysics Modeling USING COMSOL [Текст] / Roger W. Pryor. -- LLC: Jones and Bartlett Publishers. -- 2011. -- 871 с.
	
	\bibitem{Sherwood2012} Sherwood, L. Fundamentals of Human Physiolog [Текст] / L. Sherwood. — Belmon: Brooks/Cole -- 2012. -- 720 с.
	
	\bibitem{biryulinCOmsol} Бирюлин, Г.В. Теплофизический расчет в конечно-элементном пакете COMSOL/FEMLAB [Текст] / Г.В. Бирюлин. -- Методическое пособие. -- СПб: СПбГУИТМО -- 2006. -- 75 с.
\end{thebibliography}

\newpage
\addcontentsline{toc}{section}{Приложение А} 
\begin{center} 
    \textbf{Приложение А}
\end{center} 
\vspace{8mm}
\par
Листинг А.1 -- Подпись
\vspace{8mm}
\large
\begin{verbatim}
class HelloWorld
{
    MessageBox.Show("Hello world!");         
}
\end{verbatim}
\vspace{8mm}
\Large

\end{document}